%%% Research Diary - Entry
%%% Template by Mikhail Klassen, April 2013
%%% 

% arara: pdflatex: { draft: true }
% arara: makeglossaries
% arara: pdflatex: { synctex: true }    
% arara: pdflatex: { synctex: true }  

% %titlesec subcaption subfigure
%\documentclass[11pt,letterpaper]{article}
\usepackage[spanish, english]{babel} % Manejo de idiomas
\usepackage{titlesec} % para anidar secciones
%\usepackage{subfiles}
%\usepackage[hidelinks,pagebackref,backref=page,linktocpage]{hyperref} % hide links 
%\usepackage[pagebackref, hypertexnames=false]{hyperref} % hide links 
%\hypersetup{linktocpage} % for links in toc. no es necesario,eliminaba otros enlaces
%\pdfstringdefDisableCommands
% \pdfstringdefDisableCommands to temporarily disable the command while the bookmark is written.
%http://www.dickimaw-books.com/cgi-bin/faq.cgi?action=view&categorylabel=glossaries
\usepackage{subcaption} % make subfigures
\usepackage{datetime}
\usepackage{cite}
%\usepackage{natbib}
%QUITADO
\usepackage[utf8]{inputenc} %caracteres de la bibliografía

\usepackage{textcase} % Makelowercase...
\usepackage[toc,page]{appendix} % appendix and toc
\usepackage[acronym,toc,style=treenoname,order=word,subentrycounter]{glossaries}
%in subsection: acrfull instead of glsentryfull
%\renewcommand{\glscaption}{\robustify{\gls}}
%\robustify{\gls}% Make \gls not fragile
%\protect
\makeindex
\makeglossaries
%\makeglossaries main_page.tex
%\newcommand{\document}{\title}

\usepackage{comment}
\usepackage{cite}


\usepackage{mathrsfs,amsmath} 
\usepackage[makeroom]{cancel}
\usepackage{soul} % para que al subrayar no se salga de los márgenes
\usepackage{amsfonts} % for the \checkmark command 
% % % Cross mark
\usepackage{pifont}
\newcommand{\crossmark}{\hspace{1pt}\ding{55}}

\usepackage{enumitem} % personalized lists
%\usepackage{subcaption}

%\usepackage{hyperref} % link to the page on the toc
\begin{comment}
\ersetup{
    colorlinks,
    citecolor=black,
    filecolor=black,
    linkcolor=black,
    urlcolor=black,
	linktocpage}
\end{comment}
\usepackage{graphicx} % includegraphics command is implemented here
\usepackage{csquotes} % in order to do citations  \blockcquote{bibid}{text}
\usepackage{amsmath}
\usepackage{mathtools}  % mathematical symbols. loads: \usepackage{amsmath}
\usepackage{amssymb}
\usepackage{tabularx} % multiple line equations (see SmithChart) http://tex.stackexchange.com/questions/33433/how-to-place-and-number-3-short-equations-in-one-line
\usepackage{steinmetz} % loads the phase function \phase
% % cleveref 
\usepackage[noabbrev]{cleveref} % noabbrev,capitalize,nameinlink
\crefformat{equation}{equation~#2#1#3}
\Crefformat{equation}{Equation~#2#1#3}
%newcommand{\workingDate}{\textsc{2013 $|$ January $|$ 01}}
%\usepackage{tabularx} % to adjust table widths
%Change pagewidth for tables
%\usepackage[showframe=true]{geometry}
%\usepackage[showframe=false]{geometry}
\usepackage{changepage}
%%\begin{adjustwidth}{-2cm}{} \end{adjustwidth}
% also
%\renewcommand{\tabcolsep}{4.6pt}
%\begin{tabular}{@{} *{21}{l} @{}} % use "@{}" suppresses whitespace at start and end of table
\newcommand{\userName}{Isabel María Villalba Jiménez}
\newcommand{\institution}{Universitat Politècnica de Catalunya}
% To add your univeristy logo to the upper right, simply
% upload a file named "logo.png" using the files menu above.

\usepackage[overload]{empheq}
\begin{comment}
		\begin{subequations}
		\begin{align}[left = \empheqlbrace\,]
		\begin{equation}
		\Delta \phi_1 \left(t\right)=\eta V_1 sin\left(\omega_m t\right)-\frac{\pi}{2}\\ 
		\end{equation}    
		\begin{equation}
		\Delta \phi_2 \left(t\right)=\eta V_2 sin\left(\omega_m t\right)\\
		\end{equation}			
		\end{align}
		\end{subequations}
\end{comment}

		
		
\sloppy % avoid exceeding right margin





		\begin{comment}
		% IMPORTANT
		\begin{align}
		E_{out}=&\left[------------\\
		&\left. --------------------\right.\\
		&\left. ------------------- \right] e^{j\omega_c t}		
		\label{eq:MZM_04}
		\end{align} 
		\end{comment}

