\documentclass[]{article}
\usepackage{geometry}
 \geometry{
 a4paper,
 total={170mm,257mm},
 left=25mm,
 right= 35mm,
 top=25mm,
 bottom= 30mm
 }

\usepackage[T1]{fontenc}
\usepackage{lmodern}
\usepackage{amssymb,amsmath}
\usepackage{ifxetex,ifluatex}
\usepackage{fixltx2e} % provides \textsubscript
% use upquote if available, for straight quotes in verbatim environments
\IfFileExists{upquote.sty}{\usepackage{upquote}}{}
\ifnum 0\ifxetex 1\fi\ifluatex 1\fi=0 % if pdftex
  \usepackage[utf8]{inputenc}
\else % if luatex or xelatex
  \ifxetex
    \usepackage{mathspec}
    \usepackage{xltxtra,xunicode}
  \else
    \usepackage{fontspec}
  \fi
  \defaultfontfeatures{Mapping=tex-text,Scale=MatchLowercase}
  \newcommand{\euro}{€}
\fi

%%% Research Diary - Entry
%%% Template by Mikhail Klassen, April 2013
%%% 

% arara: pdflatex: { draft: true }
% arara: makeglossaries
% arara: pdflatex: { synctex: true }    
% arara: pdflatex: { synctex: true }  

% %titlesec subcaption subfigure
%\documentclass[11pt,letterpaper]{article}
\usepackage[spanish, english]{babel} % Manejo de idiomas
\usepackage{titlesec} % para anidar secciones
%\usepackage{subfiles}
%\usepackage[hidelinks,pagebackref,backref=page,linktocpage]{hyperref} % hide links 
%\usepackage[pagebackref, hypertexnames=false]{hyperref} % hide links 
%\hypersetup{linktocpage} % for links in toc. no es necesario,eliminaba otros enlaces
%\pdfstringdefDisableCommands
% \pdfstringdefDisableCommands to temporarily disable the command while the bookmark is written.
%http://www.dickimaw-books.com/cgi-bin/faq.cgi?action=view&categorylabel=glossaries
\usepackage{subcaption} % make subfigures
\usepackage{datetime}
\usepackage{cite}
%\usepackage{natbib}
%QUITADO
\usepackage[utf8]{inputenc} %caracteres de la bibliografía

\usepackage{textcase} % Makelowercase...
\usepackage[toc,page]{appendix} % appendix and toc
\usepackage[acronym,toc,style=treenoname,order=word,subentrycounter]{glossaries}
%in subsection: acrfull instead of glsentryfull
%\renewcommand{\glscaption}{\robustify{\gls}}
%\robustify{\gls}% Make \gls not fragile
%\protect
\makeindex
\makeglossaries
%\makeglossaries main_page.tex
%\newcommand{\document}{\title}

\usepackage{comment}
\usepackage{cite}


\usepackage{mathrsfs,amsmath} 
\usepackage[makeroom]{cancel}
\usepackage{soul} % para que al subrayar no se salga de los márgenes
\usepackage{amsfonts} % for the \checkmark command 
% % % Cross mark
\usepackage{pifont}
\newcommand{\crossmark}{\hspace{1pt}\ding{55}}

\usepackage{enumitem} % personalized lists
%\usepackage{subcaption}

%\usepackage{hyperref} % link to the page on the toc
\begin{comment}
\ersetup{
    colorlinks,
    citecolor=black,
    filecolor=black,
    linkcolor=black,
    urlcolor=black,
	linktocpage}
\end{comment}
\usepackage{graphicx} % includegraphics command is implemented here
\usepackage{csquotes} % in order to do citations  \blockcquote{bibid}{text}
\usepackage{amsmath}
\usepackage{mathtools}  % mathematical symbols. loads: \usepackage{amsmath}
\usepackage{amssymb}
\usepackage{tabularx} % multiple line equations (see SmithChart) http://tex.stackexchange.com/questions/33433/how-to-place-and-number-3-short-equations-in-one-line
\usepackage{steinmetz} % loads the phase function \phase
% % cleveref 
\usepackage[noabbrev]{cleveref} % noabbrev,capitalize,nameinlink
\crefformat{equation}{equation~#2#1#3}
\Crefformat{equation}{Equation~#2#1#3}
%newcommand{\workingDate}{\textsc{2013 $|$ January $|$ 01}}
%\usepackage{tabularx} % to adjust table widths
%Change pagewidth for tables
%\usepackage[showframe=true]{geometry}
%\usepackage[showframe=false]{geometry}
\usepackage{changepage}
%%\begin{adjustwidth}{-2cm}{} \end{adjustwidth}
% also
%\renewcommand{\tabcolsep}{4.6pt}
%\begin{tabular}{@{} *{21}{l} @{}} % use "@{}" suppresses whitespace at start and end of table
\newcommand{\userName}{Isabel María Villalba Jiménez}
\newcommand{\institution}{Universitat Politècnica de Catalunya}
% To add your univeristy logo to the upper right, simply
% upload a file named "logo.png" using the files menu above.

\usepackage[overload]{empheq}
\begin{comment}
		\begin{subequations}
		\begin{align}[left = \empheqlbrace\,]
		\begin{equation}
		\Delta \phi_1 \left(t\right)=\eta V_1 sin\left(\omega_m t\right)-\frac{\pi}{2}\\ 
		\end{equation}    
		\begin{equation}
		\Delta \phi_2 \left(t\right)=\eta V_2 sin\left(\omega_m t\right)\\
		\end{equation}			
		\end{align}
		\end{subequations}
\end{comment}

		
		
\sloppy % avoid exceeding right margin





		\begin{comment}
		% IMPORTANT
		\begin{align}
		E_{out}=&\left[------------\\
		&\left. --------------------\right.\\
		&\left. ------------------- \right] e^{j\omega_c t}		
		\label{eq:MZM_04}
		\end{align} 
		\end{comment}


\usepackage{url}
%\usepackage{underscore}
\usepackage[square, comma, numbers, sort]{natbib}
\bibliographystyle{unsrtnat}


% use microtype if available
\IfFileExists{microtype.sty}{\usepackage{microtype}}{}
\ifxetex
  \usepackage[setpagesize=false, % page size defined by xetex
              unicode=false, % unicode breaks when used with xetex
              xetex]{hyperref}
\else
  \usepackage[unicode=true]{hyperref}
  %\usepackage{hyperref}	
\fi
\hypersetup{breaklinks=true,
            bookmarks=true,
            pdfauthor={},
            pdftitle={},
            colorlinks=true,
            citecolor=blue,
            urlcolor=blue,
            linkcolor=magenta,
            pdfborder={0 0 0}}
\urlstyle{same}  % don't use monospace font for urls
\setlength{\parindent}{0pt}
\setlength{\parskip}{6pt plus 2pt minus 1pt}
\setlength{\emergencystretch}{3em}  % prevent overfull lines
\setcounter{secnumdepth}{2}


\author{}
\date{}

% Set roman section numbering
\renewcommand{\thesection}{\Roman{section}.} 
\renewcommand{\thesubsection}{\thesection \Roman{subsection}}


\newcommand{\competition}{The Marinexplore and Cornell University Whale Detection Challenge}
\newcommand{\imagenet}{ImageNet Large Scale Visual Recognition Competition (ILSVRC)}
\newcommand{\copyrighting}{“Copyright © 2011 by Cornell University and the Cornell Research Foundation, Inc. All Rights Reserved”}
\title{Capstone Project: \\ Right Whale call recognition using \\ Convolutional Neural Networks}\label{capstone-project}



\begin{document}
\maketitle

\section*{Machine Learning Engineer Nanodegree}\label{machine-learning-engineer-nanodegree}

Isabel María Villalba Jiménez \\ \today

\section{Definition}\label{i.-definition}
%\emph{(approx. 1-2 pages)}

\subsection{Project Overview}\label{project-overview}

%In this section, look to provide a high-level overview of the project in layman's terms. Questions to ask yourself when writing this section: - \emph{Has an overview of the project been provided, such as the problem domain, project origin, and related datasets or input data?} - \emph{Has enough background information been given so that an uninformed reader would understand the problem domain and following problem statement?}

Right whales are one of the most endangered species around the world, with only a few 400 remaining. Many of casualties among them are caused by crashing into boats. One way of avoiding these collisions is to alert ships when whales are detected in the proximity.

In order to do so, Cornell University's Bioacoustic Research Program, which has extensive experience in identifying endangered whale species, has deployed a 24/7 buoy network to guide ships from colliding with the world's last 400 North Atlantic right whales (see figure \ref{img:setup}).

\begin{figure}[htpb!]
\centering
\includegraphics[width= 0.6\textwidth]{images/kaggle_whale_detection_3}
\caption{Illustration of the deployment of the buoys in the sea while coexisting with whales \cite{kagglewhale} \label{img:setup}}
\end{figure}  

This work comes from a proposal of the Cornell University's Bioacoustic Research Program of finding new ways of improving the detection of these mammals through the audio signal of the buoys network.  Cornell University provides for the competition a dataset with recordigs made by the buoys. The proposal was made through a Kaggle competition named \href{https://www.kaggle.com/c/whale-detection-challenge}{\competition} \cite{kagglewhale} \copyrighting.

\subsection{Problem Statement}\label{problem-statement}

\begin{comment} 
In this section, you will want to clearly define the problem that you are trying to solve, including the strategy (outline of tasks) you will use to achieve the desired solution. You should also thoroughly discuss what the intended solution will be for this problem. Questions to ask yourself when writing this section: - \emph{Is the problem statement clearly defined? Will the reader understand what you are expecting to solve?} - \emph{Have you thoroughly discussed how you will attempt to solve the problem?} - \emph{Is an anticipated solution clearly defined?  Will the reader understand what results you are looking for?}
\end{comment}
Right whales make a half-dozen types of sounds, but the most characteristic one is the up-call. This type of "contact call",  is a little like small talk-- the sound of a right whale going about its day and letting others know it is nearby. In figure \ref{img:upcall} it is represented the spectrogram of an up-call which sounds like a deep, rising “whoop” that lasts about a second (sound in \cite{CornellWeb}, other calls in \cite{CornellWeb2}).

\begin{figure}[htpb!]
\centering
\includegraphics[width= 0.3\textwidth]{images/sound_upcall_quiet.jpg}
\caption{Spectrogram of a rigth whale up-call \cite{CornellWeb} \label{img:upcall}}
\end{figure}
The goal of this work is to present a model capable of detecting the right whale's up-call, which is the most characteristic call of this specie, from the audio detected by the buoys deployed in the sea.

Impressed by the working principle of Convolutional Neural Networks, I decided looking for uses beyond pure image classification. I also had been wondering if anything related to animals and whales could be done. I started looking in the internet and found several Kaggle competitions: one on whale detection through images (\href{https://www.kaggle.com/c/noaa-right-whale-recognition}{Right Whale Recognition}), and other on recognizing the North Atlantic Right Whale call (\href{https://www.kaggle.com/c/whale-detection-challenge}{\competition}). Searching for applications of Convolutional Neural Networks in sound recognition, I found an entry related to the \competition. In \cite{Nouriblog} Daniel Nouri proposed to use ConvNets not just to go across the spectrogram of the whale calls, but try to recognize a pattern by simply looking at its image, like a human could. With this proposal he got pretty good results with a very straight forward approach. I decided to give it a try and look for most used ConvNets schemes and see their performance in this competition.

The \textbf{workflow} can be organized as follows:
\begin{enumerate}
	\item sound samples exploration
	\item spectrogram generation and image processing (contrast, appropriate dimensions...)
	\item separation of dataset into training, cross-validation and test dataset and save into pickle 
	\item select ConvNets model and adjust parameters
	\begin{itemize}
		\item define structure of the ConvNet adequate for the images: depth of layers, and stride and patch size of filters and pooling layers
		\item AUC vs epochs (or training iterations), Error vs epochs, for different batch sizes
		\item tune the model using regularization and decaying learning rate		
	\end{itemize}
	\item compare the performance of winning model using the reduced version train and test dataset extracted form the train dataset
\end{enumerate}

\subsection{Metrics}\label{metrics}
\begin{comment} 
In this section, you will need to clearly define the metrics or
calculations you will use to measure performance of a model or result in your project. These calculations and metrics should be justified based on the characteristics of the problem and problem domain. Questions to ask yourself when writing this section: - \emph{Are the metrics you've chosen to measure the performance of your models clearly discussed and defined?} - \emph{Have you provided reasonable justification for the metrics chosen based on the problem and solution?}
\end{comment}

The main evaluation metric for this project will be that used in the Kaggle competition, this is the \textbf{Area Under the Curve (AUC)}, where the Curve is the ROC curve.

The \textbf{receiver operating characteristic (ROC)} curve is a graphical plot that illustrates the performance of a binary classifier system as its discrimination threshold is varied. The curve is created by plotting the true positive rate (TPR) against the false positive rate (FPR) at various threshold settings.

The true-positive rate is also known as sensitivity, recall or probability of detection. The false-positive rate is also known as the fall-out or probability of false alarm. 

The ROC curve is thus, the sensitivity as a function of fall-out. In general, if the probability distributions for both detection and false alarm are known, the ROC curve can be generated by plotting the cumulative distribution function (area under the probability distribution from $-\infty$  to the discrimination threshold) of the detection probability in the y-axis versus the cumulative distribution function of the false-alarm probability in x-axis (see figure \ref{img:ROC})\cite{wikiROC}

\begin{figure}[htpb!]
\centering
\includegraphics[width= 0.4\textwidth]{images/ROCfig}
\caption{ROC curve graphic explanation \cite{wikiwand} \label{img:ROC}}
\end{figure}

Another important measure can be the \textbf{error rate} vs iterations for a different batch sizes. The error rate used can be the percentage of wrong classified samples.

It can be also interesting to use the \textbf{confusion matrix}, which is a more detailed version of the ROC curve. The confusion matrix is a table that shows the predicted labels for each of the true input labels.

\section{Analysis}\label{ii.-analysis}
%\emph{(approx. 2-4 pages)}

\subsection{Data Exploration}\label{data-exploration}
\begin{comment}
In this section, you will be expected to analyze the data you are using for the problem. This data can either be in the form of a dataset (or datasets), input data (or input files), or even an environment. The type of data should be thoroughly described and, if possible, have basic statistics and information presented (such as discussion of input features or defining characteristics about the input or environment).
Any abnormalities or interesting qualities about the data that may need to be addressed have been identified (such as features that need to be transformed or the possibility of outliers). Questions to ask yourself when writing this section: - \emph{If a dataset is present for this problem, have you thoroughly discussed certain features about the dataset? Has a data sample been provided to the reader?} - \emph{If a dataset is present for this problem, are statistics about the dataset calculated and reported? Have any relevant results from this calculation been discussed?} - \emph{If a dataset is \textbf{not} present for this problem, has discussion been made about the input space or input data for your problem?} - \emph{Are there any abnormalities or characteristics about the input space or dataset that need to be addressed? (categorical variables, missing values, outliers, etc.)}
\end{comment}

The dataset used comes from the competition and consists of 30,000 training samples and 54,503 testing samples. Each candidate is a 2-second .aiff sound clip with a sample rate of 2 kHz. The file "train.csv" gives the labels for the train set. Candidates that contain a right whale call have label=1, otherwise label=0. These clips contain any mixture of right whale calls, non-biological noise, or other whale calls \cite{CornellWeb, CornellWeb2}. 

The training dataset is imbalanced, consisting of approximately 7000 Right Whales samples and 23000 non right whales samples (figure \ref{img:train_dataset}). In order to deal with this problem, I could just balance the samples used (get the same amount of labels from each type) or try to use an algorithm that penalizes this imbalance.

\begin{figure}[htpb!]
\centering
\includegraphics[width= 0.65\textwidth]{./images/2_dataset}
\caption{Dataset distribution \label{img:train_dataset}}
\end{figure}


\subsection{Exploratory Visualization}\label{exploratory-visualization}
\begin{comment}
In this section, you will need to provide some form of visualization
that summarizes or extracts a relevant characteristic or feature about the data. The visualization should adequately support the data being used. Discuss why this visualization was chosen and how it is relevant.
Questions to ask yourself when writing this section: - \emph{Have you visualized a relevant characteristic or feature about the dataset or input data?} - \emph{Is the visualization thoroughly analyzed and discussed?} - \emph{If a plot is provided, are the axes, title, and datum clearly defined?}

- Compare spectrogram of some non whale to some whales and then explain why I trimmed it
\end{comment}
The audio is recorded after the buoys auto-detect the characteristic up-call, what biases the dataset to that kind of calls. Thus, it makes sense to detect only the up-call, which is the most characteristic and most frequently emitted call (details on the deployment of the buoys and how the recordings are made in \cite{McDonald2002}). 

In figure \ref{img:samples}, samples corresponding to right-whale up-call (label = 1) show the that the energy of the signal is below 250Hz (see figure \ref{img:upcall}) and they exhibit a clear pattern. This fact will help to reduce information processed to that range of frequencies. 
In the second row, corresponding to negative identifications (label = 0) this pattern is not present. However, some samples from from other whale species or corresponding to right-whale making other calls are included in the label 0. This could be the case of sample train6776 in figure \ref{img:samples}.

Recordings have a duration of 1.8 seconds and a sampling rate of 2000 Hz. The raw spectrogram of the recordings result in images of 129x23 pixels.

\begin{figure}[htpb!]
\centering
\includegraphics[width= \textwidth]{./images/2_samples}
\caption{Samples for right whale up-call (label 1) and no-right-whale up-call (label 0).  \label{img:samples}}
\end{figure} 


\subsection{Algorithms and Techniques}\label{algorithms-and-techniques}
\begin{comment}
In this section, you will need to discuss the algorithms and techniques you intend to use for solving the problem. You should justify the use of each one based on the characteristics of the problem and the problem domain. Questions to ask yourself when writing this section: - \emph{Are the algorithms you will use, including any default variables/parameters in the project clearly defined?} - \emph{Are the techniques to be used thoroughly discussed and justified?} - \emph{Is it made clear how the input data or datasets will be handled by the algorithms and techniques
chosen?}
-- Maybe : logistic and SVM?
\end{comment}
In order to perform the prediction I will try to implement well-known and widely-implemented models of ConvNets.

Convolutional Neural Networks (ConvNets) are a type of Neural Networks (NNs) that make the assumption that inputs are images. This allows to encode certain properties into the architecture that make the forward function more efficient to implement and vastly reduce the amount of parameters in the network \cite{cs231convnets}.
ConvNets, like other NNs, are made up of layers. These layers (called hidden layers) transform input 3D volumes to output 3D volumes with some differentiable function that may or may not have parameters \cite{cs231convnets}. This is an interesting property of ConvNets: layers have neurons arranged in 3 dimensions (width, height and depth) (see figure \ref{img:cnn}). %Each hidden layer is made up of neurons, where each neuron is fully connected to all neurons in the previous layer, and where neurons in a single layer function completely independently and do not share any connections. The last fully-connected layer is called the “output layer” and in classification settings it represents the class scores. 

There are three main types of layers that are used to build ConvNets \cite{cs231convnets}: % Convolutional Layer, Pooling Layer, and Fully-Connected Layer .
\begin{itemize}
	\item Convolutional Layer (CONV): computes the output of neurons connected to local regions of the input, performing the dot product between their weight and a small region that are connected to. If the input is 3D (i.e. an image with RGB colors) this layer will also be 3-dimensional. Four hyperparameters control the size of the output volume: the depth, filter size, stride and zero-padding. Depth (D) is related to the number of filters used in the layer, the filter or patch have dimensions (F) (i.e. 2x2) and stride (S) is referred to the displacement of the filter. The amount Zero-padding (P), which allows to control the spatial size of outputs.
	\item Pooling Layer (POOL) (or Subsampling Layer): performs downsampling operation along spatial dimensions. Can be max-pooling (taking the maximum of a region), average-pooling (taking the average of a region) or other types of results from applying a function to a region of the image. It has two main hyperparameters: the spatial extent of the filter or patch where pooling is applied (F) and the stride (S) of this filter.
	\item Fully-Connected Layer (FC): is the same type of layer as in NNs.
\end{itemize}

Usually a CONV layer is followed by a RELU layer, which perform an element-wise activation function (i.e. max(0,x), logistic function, tanh).

\begin{figure}[htpb!]
\centering
\includegraphics[width= 0.5\textwidth]{images/cnn}
\caption{ConvNet 3D structure \cite{cs231convnets}\label{img:cnn}}
\end{figure}

LeNet is one of the first successful applications of Convolutional Networks, developed by Yann LeCun in the 90s. One of the versions of LeNet is LeNet-5, which is highly used for handwritten and machine-printed character recognition. Figure \ref{img:lenet5} shows the structure of the network, composed of 2 convolutional layers, 2 fully connected layers and 2 subsampling or pooling layers. The layers follow the sequence: INPUT -> CONV -> RELU -> SUBS (POOL) -> CONV -> RELU -> SUBS (POOL) -> FC -> RELU -> FC, for INPUT, CONV, RELU, SUBS, FC. % meaning respectively the Input, Convolutional, RELU, Subsampling (or max pooling) and Fully Connected layer.
%\pagebreak
\begin{figure}[htpb!]
\centering
\includegraphics[width= 0.99\textwidth]{images/lenet5}
\caption{LeNet-5 structure \cite{Lecun98} \label{img:lenet5}}
\end{figure}

Other more complex ConvNet is AlexNet, developed by Alex Krizhevsky et al. \cite{Krizhevsky12}. The AlexNet was submitted to the ImageNet ILSVRC challenge in 2012 and significantly outperformed the second runner-up (top 5 error of 16\% compared to runner-up with 26\% error). It has a very similar architecture to LeNet, but it is  deeper, bigger, and features Convolutional Layers stacked on top of each other (previously it was common to only have a single CONV layer always immediately followed by a POOL layer)\cite{cs231convnets}. Figure \ref{img:imagenet12} shows the structure of the network, composed of 5 convolutional layers, 3 fully connected layers and 3 subsampling or pooling layers. The layers follow the sequence: INPUT -> CONV -> RELU -> SUBS -> CONV -> RELU -> SUBS -> CONV -> RELU -> CONV -> RELU -> SUBS -> FC -> RELU -> FC. %, for INPUT, CONV, RELU, SUBS, FC meaning respectively the Input, Convolutional, RELU, Subsampling (or max pooling) and Fully Connected layer .% Cambiar, LITERAL!


\begin{figure}[htpb!]
\centering
\includegraphics[width= 0.99\textwidth]{images/imagenet12}
\caption{AlexNet structure\cite{Krizhevsky12} \label{img:imagenet12}}
\end{figure}

Some parameters need adjustment to adequate the network to this work, like the size of the patch (F) and stride (S) of the convolutional and pooling layers, and also the depth (D) of each convolutional layer. %This issue will be discussed in later sections.

\subsection{Benchmark}\label{benchmark}
\begin{comment}
In this section, you will need to provide a clearly defined benchmark result or threshold for comparing across performances obtained by your solution. The reasoning behind the benchmark (in the case where it is not an established result) should be discussed. Questions to ask yourself when writing this section: - \emph{Has some result or value been provided that acts as a benchmark for measuring performance?} - \emph{Is it clear how this result or value was obtained (whether by data or by hypothesis)?}
\end{comment}

I will try to compare the performance of popular ConvNets (i.e. LeNet-5 proposed by Lecun \cite{Lecun98} or AlexNet, the winner of the 2010 and 2012 \imagenet \, proposed by Krizhevsky \cite{Krizhevsky12, Krizhevsky2010}) with the performance of the winning model of the competition which is based on Gradient Boosting (SluiceBox: \href{https://github.com/nmkridler/moby}{Github}) and the Daniel Nouri's model based on Krizhevsky's 2012 ILSVRC ConvNet model \cite{Krizhevsky12} (\href{https://speakerdeck.com/dnouri/practical-deep-neural-nets-for-detecting-marine-mammals/}{source}), which first inspired this work.

The Area Under the Curve (AUC) (see the Evaluation metrics section \ref{evaluation-metrics}) of these models in the public leaderboard was:

\begin{itemize}
	\item SluiceBox: 0.98410 (1st position)
	\item Nouri: 0.98061 (6th position with 1/4 times the submission of the winner)
\end{itemize}

Nevertheless, I will not be able compare the performance of my models to these results. The reason is that I do not have the test labels and also, the public leaderboard data test used is slightly different for each participant. I will try two different approaches:

\begin{enumerate}
	\item assuming that there are enough complete data samples (train dataset), trying to increase the accuracy as much as possible (this will be the main option)
	\item assuming the predictions generated by the winning model as test labels and them as reference to compare our model with
\end{enumerate}
\begin{comment}

However, as the competition is closed, I cannot access the test labels and compare the performance to others algorithms. Two options sound plausible:
\begin{itemize}
	\item divide the train dataset into: train, cross-validation and test dataset
	\item use the winning algorithm to generate an estimation of the test labels and so increase the amount of data for training.
\end{itemize}
\end{comment} 
%pero no podemos comparar directamente porque no tenemos el test dataset

\section{Methodology}\label{iii.-methodology}
%\emph{(approx. 3-5 pages)}

\subsection{Data Preprocessing}\label{data-preprocessing}
\begin{comment}
In this section, all of your preprocessing steps will need to be clearly documented, if any were necessary. From the previous section, any of the
abnormalities or characteristics that you identified about the dataset
will be addressed and corrected here. Questions to ask yourself when
writing this section: - \emph{If the algorithms chosen require
preprocessing steps like feature selection or feature transformations,
have they been properly documented?} - \emph{Based on the \textbf{Data
Exploration} section, if there were abnormalities or characteristics
that needed to be addressed, have they been properly corrected?} -
\emph{If no preprocessing is needed, has it been made clear why?}
\end{comment}
The processing of the images will consist in the usual mean subtraction and normalization (division by standard deviation). The result of this process applied to images in figure \ref{img:samples} is presented in figure \ref{img:samples_unprocessed}. 

\begin{figure}[htpb!]
\centering
\includegraphics[width= \textwidth]{./images/2_samples_unprocessed}
\caption{ Samples after mean subtraction and division by standard deviation for right whale up-call (label 1) and no-right-whale up-call (label 0).  \label{img:samples_unprocessed}}
\end{figure} 

However, after applying this processing, images are not clear enough and energy where up-calls are contained is very low. In order to make them more visible, images in figure \ref{img:samples} are first processed applying log10 and then normalized subtracting mean and dividing by standard deviation.

\begin{figure}[htpb!]
\centering
\includegraphics[width= \textwidth]{./images/2_samples_processed}
\caption{Processed samples after mean subtraction and normalization for right whale up-call (label 1) and no-right-whale up-call (label 0).  \label{img:samples_processed}}
\end{figure} 

As it was commented previously, recordings have a duration of 1.8 seconds and a sampling rate of 2000Hz. The raw spectrogram of the recordings result in images of 129x23 pixels. Two problems arise from this: the excess of redundant information in frequencies not important for the detection of the up-call and the limitation in terms of the image of being processed by the network.

In order to solve the first problem, the frequency range will be limited 0-250 Hz, what will result in images of size 33x23 px. Still, this is not good for a ConvNet with lots of convolutions and pooling. It is necessary to resize the image and a good number is multiple of 2. Hence, I will choose images to be 32x32 px and will obtain additional pixels using interpolation. %decir qué tipo de interpolación

\begin{figure}[htpb!]
\centering
\includegraphics[width= \textwidth]{./images/2_samples_cropped}
\caption{Samples reduced to 32x32 pixels after log10 and mean subtraction and normalization for right whale up-call (label 1) and no-right-whale up-call (label 0).  \label{img:samples_processed}}
\end{figure} 


\subsection{Implementation}\label{implementation}
\begin{comment}
In this section, the process for which metrics, algorithms, and
techniques that you implemented for the given data will need to be
clearly documented. It should be abundantly clear how the implementation
was carried out, and discussion should be made regarding any
complications that occurred during this process. Questions to ask
yourself when writing this section: - \emph{Is it made clear how the
algorithms and techniques were implemented with the given datasets or
input data?} - \emph{Were there any complications with the original
metrics or techniques that required changing prior to acquiring a
solution?} - \emph{Was there any part of the coding process (e.g.,
writing complicated functions) that should be documented?}

            depth_1 = 6
            depth_2 = 6
            depth_3 = 16
            depth_4 = 16
            depth_5 = 120

            patch_size_1 = 5
            patch_size_2 = 2
            patch_size_3 = 6
            patch_size_4 = 2
            patch_size_5 = 6
            

In the training phase of the CNN where we estimate the CNV and FC layer filter taps, we use stochastic gradient descent algorithm [7]. Training is done offline through backward propagation algorithms [8]. We have built the CNV using Tensor Flow [9] as provided part of this submission.
Gradient descent algorithms are very popular in solving quadratic equations though dynamic programming. Stochastic gradient descent algorithm is a variant of gradient descent where we use a small random sample of data at every iteration in order to find optimum weights of the CNN.
One of the important parameters of gradient descent is the learning rate which is typically represented by α. Basically, large α helps gradient descent to converge faster but with decent quality of weights, while
small α has lower converge speed but achieves better weight. Thus, we have changed the α parameter along the training with certain decay factor. We start with large α value and reduce it as we iterate over the dataset. This has improved the accuracy by couple of percent compared to a fix α value.

\end{comment}
The model implemented is LeNet-5 (see figure \ref{img:lenet5}). The only difference in the F6 layer, which has been removed. This layer is used for the detection of ASCII characters in 7x12 bitmaps, but this network does not intend to do so, it just needs to detect right-whale up-call or not.

The structure of the network is as follows (F= Filter, S= Stride, D= Depth):
\begin{itemize}
	\item INPUT layer: 32x32 px image in gray scale represented with 8-bit number (0-255 levels).
	\item C1 - CONV layer: F = 5x5, S = 1, D = 6
	\item S2 - POOL layer: F = 2x2, S = 2, D = 6
	\item C3 - CONV layer: F = 5x5, S = 1, D = 16
	\item S4 - POOL layer: F = 2x2, S = 2, D = 16
	\item C5 - CONV layer: F = 5x5, S = 1, D = 120
	\item F5 - FC layer: neurons= 120 x 2, one per label(considering label 0 and label 1)
\end{itemize}

The training is performed using mini-batch gradient descent, which is a version of the true gradient descent (combines batch and stochastic gradient descent), used when data amount is quite high. It iterates over batches of n samples in order to approach the minimum of the cost function step by step (epochs). Mini-batch gradient descent reduces the variance of the parameter updates, which can lead to more stable convergence. It also can make use of highly optimized matrix optimizations common to state-of-the-art deep learning libraries that make computing the gradient with respect to a mini-batch very efficient \cite{ruderweb}.

Gradient descent main parameter is the learning rate ($\alpha$). The learning rate express the speed of convergence of the gradient descent. Large learning rates lead to faster convergence but it may miss the minimum and not converge properly. Low learning rates lead to a better convergence point, but it is slower, requiring more steps and more memory allocation. A good compromise is to use a decaying learning rate: high values for the first epochs to accelerate the initial convergence and then smaller ones to slow it down. 

Regularization is a common way to prevent over-fitting and the most used method is L2 regularization. This type of regularization penalizes the square magnitude of all parameters, adding the term $\frac{1}{2}\lambda \omega ^2$ to the prediction, for $\lambda$ the regularization strength \cite{cs231convnets}. In this work L2 regularization will be used to control the over-fitting.

The network has been implemented making use of Tensorflow, using the basic of ConvNets explained in the Udacity Deep Learning Course \cite{deepgithub} and then extending the functionality to more complex networks.

\subsection{Refinement}\label{refinement}
\begin{comment}
In this section, you will need to discuss the process of improvement you made upon the algorithms and techniques you used in your implementation.
For example, adjusting parameters for certain models to acquire improved solutions would fall under the refinement category. Your initial and final solutions should be reported, as well as any significant intermediate results as necessary. Questions to ask yourself when writing this section: - \emph{Has an initial solution been found and clearly reported?} - \emph{Is the process of improvement clearly documented, such as what techniques were used?} - \emph{Are intermediate and final solutions clearly reported as the process is improved?}
\end{comment}
\newcommand{\lr}{0.0005}
The process followed has been one the iteration over different parameters to obtain the best combination. Figure x shows there is little over fitting, since validation and test curves follow the training curve.
\subsubsection{Select batch-size}
In order to select the proper batch size simulations have been performed for different learning rates. It can be seen that for the same learning rate an a batch size of X reaches a greater value.

After many simulations, I have observed that there must be a trade-off between batch-size and instability. The bigger the batch, the stabler the curves, but the poorer the performance since the number of epochs is smaller. The batch size must be big enough to provide a less noisy curve but small enough to give good values of prediction.

From figure \ref{img:batch_comparasion} it seems that a good compromise value is a batch size of 5.

\begin{figure}[htpb!]
\centering
\includegraphics[width = 0.6\textwidth]{{images/learning_rate_\lr0_val_error_AWS_val_12_}.png}
\caption{Error curve for the validation dataset, for different batch sizes and a learning rate $\alpha$=\lr  \label{img:batch_comparasion}}
\end{figure} 





\subsubsection{Select learning rate}
Secondly, for a fixed batch size , the cost has been calculated for different learning rates. Learning rates too big may not find the minimum and converge too fast. Small learning rate may be too slow and not fast enough for small dataset like the one in this work. A compromise value must be chosen, showing a slope good that guarantees convergence. Figure \ref{img:learning_rate} shows that for the selected batch size of 5, a learning rate $\alpha$ of 0.0005  seems like a good value. Figure \ref{img:batch} shows that there is not over-fitting, since there is no gap between the training curve and the validation and test curves.
\begin{figure}[htpb!]
\centering
\includegraphics[width = 0.6\textwidth]{{images/batch_size_5_val_auc_AWS_val_08}.png}
\caption{AUC for the validation dataset, for different learning rates and a batch size = 5  \label{img:learning_rate}}
\end{figure} 

\begin{figure}[htpb!]
\centering
\includegraphics[width = 0.6\textwidth]{{images/learning_rate_\lr0_val_auc}.png}
\caption{AUC for batch size = 5 and learning rate $\alpha$=\lr   \label{img:batch}}
\end{figure} 

\subsubsection{Select regularization}
Finally, for fixed batch size and learning rate, the AUC has been calculated for regularization parameters. Regularization is a good way to limit over-fitting, allowing the model to generalize better. Figure \ref{img:regularization} shows the AUC curve for different values of the regularization parameters. Figure \ref{img:regularization_1} shows that a good value for the regularization parameter can is near 0.95, and figure \cref{img:regularization_2} shows that for the selected batch size of 5 and learning rate of 0.0005 a good regularization value is 0.97, since it allows to reach a greater AUC.

\begin{figure}[htpb!]
	\centering

    \begin{subfigure}[l]{0.5\textwidth}
		%\centering
		\includegraphics[width=\linewidth]{{images/batch_size_5_lr_\lr00_val_auc_AWS_reg_0.9_to_0.01}.png}
		\caption{\label{img:regularization_1}}
    \end{subfigure}%
    ~
    \begin{subfigure}[r]{0.5\textwidth}
    		%\centering
    		\includegraphics[width=\linewidth]{{images/batch_size_5_lr_\lr00_val_auc_AWS_val_09}.png}
    		\caption{  \label{img:regularization_2}}
    \end{subfigure}%
    \caption{AUC in the validation dataset for different regularization values, batch size = 5 and learning rate $\alpha$=\lr \ref{img:regularization}}
\end{figure} 

%\subsubsection{Dropout and Batch normalization}
%\pagebreak
\section{Results}\label{iv.-results}

%\emph{(approx. 2-3 pages)}

\subsection{Model Evaluation and Validation}\label{model-evaluation-and-validation}
\begin{comment}
In this section, the final model and any supporting qualities should be evaluated in detail. It should be clear how the final model was derived and why this model was chosen. In addition, some type of analysis should be used to validate the robustness of this model and its solution, such as manipulating the input data or environment to see how the model's solution is affected (this is called sensitivity analysis). Questions to ask yourself when writing this section: - \emph{Is the final model reasonable and aligning with solution expectations? Are the final parameters of the model appropriate?} - \emph{Has the final model been tested with various inputs to evaluate whether the model generalizes well to unseen data?} - \emph{Is the model robust enough for the problem? Do small perturbations (changes) in training data or the input space greatly affect the results?} - \emph{Can results found from the model be trusted?}
\end{comment}
The model used is the LeNet-5 \cite{Lecun98}. In figure X it is presented the error for different epochs.

%In table X it is presented the confusion matrix.

In figure X it is also presented the AUC por different epochs.

One problem I faced when training the model was the limitation in terms of data samples for training which reduced the maximum AUC obtained. In order to solve it, labels for the test dataset have been obtained using the evaluations from the winning model as true labels (SluiceBox: 0.98410 AUC). This has allowed o increase the volume of data samples available to train the model and hence, increase its performance. I have taken this decision after taking into consideration the processed of obtaining of the samples. The sounds were firstly recorded when a buoy detected an up-all in the area. Afterwards, samples were labeled by human ear. This leads to a lot of mislabeled samples (as mentioned in \cite{Nouriblog}), and gives the intuition that using the labels from the prediction of a good model as test dataset, freeing samples for training, can be more good than harm.


\subsection{Justification}\label{justification}
\begin{comment}
In this section, your model's final solution and its results should be compared to the benchmark you established earlier in the project using some type of statistical analysis. You should also justify whether these results and the solution are significant enough to have solved the problem posed in the project. Questions to ask yourself when writing this section: - \emph{Are the final results found stronger than the benchmark result reported earlier?} - \emph{Have you thoroughly analyzed and discussed the final solution?} - \emph{Is the final solution significant enough to have solved the problem?}
\end{comment}
This work has shown the use of simple ConvNet for audio recognition, obtaining good performance with an AUC of up to 0.958, whereas more complex models as SluiceBox obtained 0.98410 AUC and Nuori 0.98061. 

One of the advantages of simpler networks relies in the reduction of time required for training the model, resulting in less computer requirements. With a vast dataset, the system would increase significantly the performance and equal those with more complex stucture, making it a great, simple and nice choice. %less capacity and less time for training


\section{Conclusion}\label{v.-conclusion}

%\emph{(approx. 1-2 pages)}

\subsection{Free-Form Visualization}\label{free-form-visualization}
%In this section, you will need to provide some form of visualization that emphasizes an important quality about the project. It is much more free-form, but should reasonably support a significant result or characteristic about the problem that you want to discuss. Questions to ask yourself when writing this section: - \emph{Have you visualized a relevant or important quality about the problem, dataset, input data, or results?} - \emph{Is the visualization thoroughly analyzed and discussed?} - \emph{If a plot is provided, are the axes, title, and datum clearly defined?}

%CAPAS
\renewcommand{\lr}{0.0005}
In this section I will try to give an insight of the network, presenting, as it is frequently done, the characteristics f the first layer. Figure \ref{img:weights} it is presented the evolution of the filters of the first CONV layer (F=5x5) with depth 6 (D=6). It is clear how, due to backprogagation and minimization of the cost function, weights are changed and they evolve to new and more adequate values for prediction.
Figure \ref{img:biases} shows the evolution of the biases with the number of epochs and how they tend stabilize. Finally, figure \ref{img:activation} shows the output every 500 epochs for some samples.

\begin{figure}[htpb!]
\centering
\includegraphics[width = 0.35\textwidth]{{images/weights_batch_size_5_learning_rate_\lr0}.png}
\caption{Weights of the first layer with 6 filters of depth every 500 epochs for batch size=5 and learning rate $\alpha$=\lr \label{img:weights}}
\end{figure} 

\begin{figure}[htpb!]
\centering
\includegraphics[width = 0.6\textwidth]{{images/bias_batch_size_5_learning_rate_\lr0}.png}
\caption{Biases of the first layer with 6 filters of depth for batch size=5 and learning rate $\alpha$=\lr  \label{img:biases}}
\end{figure} 



\begin{figure}[htpb!]
\centering
\includegraphics[width = 0.35\textwidth]{{images/activation_batch_size_5_learning_rate_\lr0}.png}
\caption{Activation of the first layer with 6 filters of depth every 500 epochs for batch size=5 and learning rate $\alpha$=\lr  \label{img:activation}}
\end{figure} 
\subsection{Reflection}\label{reflection}

%In this section, you will summarize the entire end-to-end problem solution and discuss one or two particular aspects of the project you found interesting or difficult. You are expected to reflect on the project as a whole to show that you have a firm understanding of the entire process employed in your work. Questions to ask yourself when writing this section: - \emph{Have you thoroughly summarized the entire process you used for this project?} - \emph{Were there any interesting aspects of the project?} - \emph{Were there any difficult aspects of the project?} - \emph{Does the final model and solution fit your expectations for the problem, and should it be used in a general setting to solve these types of problems?}
In this project, it is presented how simple ConvNets can be used to label data from the image of the spectrogram, as it in many classification problems. More complex network may achieve better results, but for an image of 32x32px, I prefer to take the strategy: the simpler, the better.

In this case, the use of LeNet-5 has given a X AUC, which is really good given the low complexity of the network. 
%Comment batch... and other parameters.
\subsection{Improvement}\label{improvement}

%In this section, you will need to provide discussion as to how one aspect of the implementation you designed could be improved. As an example, consider ways your implementation can be made more general, and what would need to be modified. You do not need to make this improvement, but the potential solutions resulting from these changes are considered and compared/contrasted to your current solution.
% Questions to ask yourself when writing this section: - \emph{Are there further improvements that could be made on the algorithms or techniques you used in this project?} - \emph{Were there algorithms or techniques you researched that you did not know how to implement, but would consider using if you knew how?} - \emph{If you used your final solution as the new benchmark, do you think an even better solution exists?}

In this work, the problem is to recognize a simple and very specific spectrogram pattern from in a form of an image. A good rule of thumb for this is the simpler, the better and LeNet-5 achieves good performance with a simple structure.  

The complexity of the model should not be the focus, since like in any Machine Learning problem, the most important factor is the amount of data. Collecting more samples and extending the training would make LeNet-5 a good competitor against other models achieving better performance with over-complicated models.

Once resolved the issue with the amount of data, more complex models are good start points to increase the performance and achieve better results and compare the trade-off between complexity with results. Good examples of models to try are AlexNet \cite{Krizhevsky12} or GoogLeNet \cite{Szegedy16}. Dropout has not been applied in this work, since the network has lots of max-pooling layers and regularization, and in these cases, the use of dropout is not clear. Introducing dropout could be a good option to try. Also, methods to make faster training, as Batch Normalization may lead to better results \cite{szegedy15batch}.
\begin{comment}

\begin{center}\rule{3in}{0.4pt}\end{center}

\textbf{Before submitting, ask yourself. . .}

\begin{itemize}
\itemsep1pt\parskip0pt\parsep0pt
\item
  Does the project report you've written follow a well-organized structure similar to that of the project template?
\item
  Is each section (particularly \textbf{Analysis} and \textbf{Methodology}) written in a clear, concise and specific fashion? Are there any ambiguous terms or phrases that need  clarification?
\item
  Would the intended audience of your project be able to understand your analysis, methods, and results?
\item
  Have you properly proof-read your project report to assure there are minimal grammatical and spelling mistakes?
\item
  Are all the resources used for this project correctly cited and referenced?
\item
  Is the code that implements your solution easily readable and properly commented?
\item
  Does the code execute without error and produce results %mac  similar to those reported?
\end{itemize}
\end{comment}
\bibliography{../../../Dropbox/PhD/library}
%\bibliography{../../../../../media/mabelvj/Darsena/DropboxUbuntu/Dropbox/PhD/library}

\end{document}
